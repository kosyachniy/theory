\documentclass{article}

% Свои операторы
\usepackage{amsmath}
\DeclareMathOperator{\tg}{tg}

%
\usepackage{natbib}
\usepackage{graphicx}

% Кодировка
\usepackage[utf8]{inputenc}

% Для написаний
\usepackage{amssymb}

% Русский язык
\usepackage[russian]{babel}

% Шрифт
\renewcommand{\familydefault}{\sfdefault}

% Графики
% \usepackage{pgfplots}
% \RequirePackage{luatex85,shellesc}

% Ссылки
\usepackage{hyperref}


\title{Математика}
\author{Полоз Алексей}


\begin{document}

\maketitle

\section{Функция}
\label{section:function}
\subsection{Понятие}
[1]\\
Функция f(x) --
\begin{itemize}
    \item Отображение из множества определения в множество значения функции
    \item  Соответствие между различными значениями аргумента x и значениями функции f(x)
\end{itemize}

$x \mapsto f(x), x \in \mathbb{R}$ 


\noindent\rule{\textwidth}{1pt}

Свойства:
\begin{itemize}
    \item Каждому аргументу соответствует только одно значение f(x)
\end{itemize}


\subsection{Область определения и значения}
[2]\\
$D(f)$ -- область определения функции (те значения x, для которых функция задана)

$E(f)$ -- область значения функции (все значения, что функции может принимать)

\noindent\rule{\textwidth}{1pt}

Примеры:
\begin{itemize}
    \item $f(x)=\frac{1}{x-1}, D(f)=\mathbb{R}\backslash\{1\}, E(f)=\mathbb{R}\backslash\{0\}$
\end{itemize}


\subsection{График}
[3]\\
<рандомная линия поверх пересечения осей x, y надпись y=f(x)>


\subsection{Непрерывность}
[4]\\
Разрывы

\begin{itemize}
    \item Одна точка выбивается
    \item Резкий скачок значения функции
    \item Разрывы с ассимптотой (прямая, к которой приближается, но не пересекает)
\end{itemize}

\noindent\rule{\textwidth}{1pt}
[8]\\

\hyperref[section:derivative]{производная $\to$}\\
Функция $f(x)$ непрерывна в точке $x=a$: $$\lim_{x \to a} f(x) = f(\lim_{x \to a} x) = f(a)$$

\subsection{Гладкость}
[5]\\
отсутствие углов

! бывает не гладкая ни в одной точке
Примеры:
Функция Веерштрасса

\noindent\rule{\textwidth}{1pt}
[9]\\

\hyperref[section:derivative]{производная $\to$}\\
Гладкие функции -- функции с непрерывной производной\\
!нестрого -- без изломов

[13]\\
\hyperref[section:tangent]{касательная $\to$}\\
Доказательство:\\
Если производная резко меняет своё значение в некоторой точке - меняется и направление касательной, получается излом.

\subsection{Секущая}
[11]\\
-- прямая, которая проходит через 2 точки на графике.

<график>\\
$y(x) = \frac{\Delta y}{\Delta x}(x - x_0) + f(x_0)$

Доказательство:\\
Проходит через точки $x_0$ и $x_0 + \Delta x$. Соответственно функция принимает значения $y_0$ и $y_0 + \Delta y$.

\subsection{Касательная}
[12]\\
\label{section:tangent}
\hyperref[section:limit]{предел $\to$}\\
Если $$\lim_{x \to 0}$$, то секущая перейдёт в касательную.

-- прямая, которая пересекает график функции в одной точке.

Пересекает его под нулевым углом.

Следствие:\\
Очень хорошо приближает график в окрестности этой точки.

\hyperref[section:derivative]{производная $\to$}\\
<график>\\
$y(x) = f^\prime (x_0)(x - x_0) + f(x_0)$


\section{Предел}
\label{section:limit}
\subsection{Функции}
[6]\\
$\lim_{}$

! Неформально:

$$\lim_{x \to a} f(x)$$ -- величина, к которой стремится $f(x)$, если $x$ стремится к $a$

\noindent\rule{\textwidth}{1pt}

Примеры:

\begin{itemize}
	\item $(1 + x)^\frac{1}{x}$

		\begin{tabular}{ | l | l | }
			\hline
			$x$ & $f(x)$ \\ \hline
			$0.1$ & $2.593...$ \\
			$0.01$ & $2.704...$ \\
			$0.001$ & $2.716...$ \\
			$0.0001$ & $2.718...$ \\
			\hline
		\end{tabular}

		$$\lim_{x \to 0} (1 + x)^\frac{1}{x} = 2.7182...$$

	\item  $\frac{1}{x}$

		\begin{tabular}{ | l | l | }
			\hline
			$x$ & $f(x)$ \\ \hline
			$0.1$ & $10$ \\
			$0.01$ & $100$ \\
			$0.001$ & $1000$ \\
			$0.0001$ & $10000$ \\
			\hline
		\end{tabular}

		$$\lim_{x \to 0} \frac{1}{x} = \infty$$

\end{itemize}


\section{Производная}
\label{section:derivative}
\subsection{Понятие}
[7]\\
-- скорость роста функции

<график>

$y = k*x + b$

$k = \frac{f(x + \Delta x) - f(x)}{\Delta x}$ -- скорость роста

$$f^\prime(x) = \lim_{\Delta x \to 0} \frac{f(x + \Delta x) - f(x)}{\Delta x}$$ -- производная функции $f(x)$ в точке $x$

\subsection{Геометрический смысл}
[14]\\
-- угловой коэффициент касательной к графику функции.

[10]\\
<график> !дополнить h, l из доказательства

$y = k * x + b$

$k = \tg \alpha$

\noindent\rule{\textwidth}{1pt}

Доказательство:\\
Для любой точки $M$ прямой $y = k x, y_0 = kx_0 + b$. По определению: $\tg \alpha = \frac{h}{l} = \frac{y_0}{x_0}$. $\rightarrow$ $k = \tg \alpha$















\newpage
\section{Обозначения}
\begin{tabular}{ | l | l | }
	\hline
	Обозначение & Значение \\ \hline
	$\to$ & Стремится \\
	\hline
\end{tabular}


\newpage
\bibliographystyle{plain}
\bibliography{references}
\end{document}
