\documentclass{article}

% Кодировка
\usepackage[utf8]{inputenc}

% Для написаний
\usepackage{amssymb}

% Русский язык
\usepackage[russian]{babel}

% Шрифт
\renewcommand{\familydefault}{\sfdefault}

% Графики
% \usepackage{pgfplots}
% \RequirePackage{luatex85,shellesc}

% Ссылки
\usepackage{hyperref}

%
\usepackage{natbib}
\usepackage{graphicx}


\title{Математика}
\author{Полоз Алексей}

\begin{document}

\maketitle

\section{Функция}
\label{section:function}
\subsection{Понятие}
Функция f(x) --
\begin{itemize}
    \item Отображение из множества определения в множество значения функции
    \item  Соответствие между различными значениями аргумента x и значениями функции f(x)
\end{itemize}

$x \mapsto f(x), x \in \mathbb{R}$ 


\noindent\rule{\textwidth}{1pt}

Свойства:
\begin{itemize}
    \item Каждому аргументу соответствует только одно значение f(x)
\end{itemize}


\subsection{Область определения и значения}
$D(f)$ -- область определения функции (те значения x, для которых функция задана)

$E(f)$ -- область значения функции (все значения, что функции может принимать)

\noindent\rule{\textwidth}{1pt}

Примеры:
\begin{itemize}
    \item $f(x)=\frac{1}{x-1}, D(f)=\mathbb{R}\backslash\{1\}, E(f)=\mathbb{R}\backslash\{0\}$
\end{itemize}


\subsection{График}
<рандомная линия поверх пересечения осей x, y надпись y=f(x)>


\subsection{Непрерывность}
Разрывы

\begin{itemize}
    \item Одна точка выбивается
    \item Резкий скачок значения функции
    \item Разрывы с ассимптотой (прямая, к которой приближается, но не пересекает)
\end{itemize}

\noindent\rule{\textwidth}{1pt}

\hyperref[section:derivative]{производная $\to$}

Функция $f(x)$ непрерывна в точке $x=a$: $$\lim_{x \to a} f(x) = f(\lim_{x \to a} x) = f(a)$$

\subsection{Гладкость}
отсутствие углов

! бывает не гладкая ни в одной точке
Примеры:
Функция Веерштрасса

\noindent\rule{\textwidth}{1pt}

\hyperref[section:derivative]{производная $\to$}

Гладкие функции -- производная которых непрерывна


\section{Предел}
\label{section:limit}
\subsection{Функции}
$\lim_{}$

! Неформально:

$$\lim_{x \to a} f(x)$$ -- величина, к которой стремится $f(x)$, если $x$ стремится к $a$

\noindent\rule{\textwidth}{1pt}

Примеры:

\begin{itemize}
	\item $(1 + x)^\frac{1}{x}$

		\begin{tabular}{ | l | l | }
			\hline
			$x$ & $f(x)$ \\ \hline
			$0.1$ & $2.593...$ \\
			$0.01$ & $2.704...$ \\
			$0.001$ & $2.716...$ \\
			$0.0001$ & $2.718...$ \\
			\hline
		\end{tabular}

		$$\lim_{x \to 0} (1 + x)^\frac{1}{x} = 2.7182...$$

	\item  $\frac{1}{x}$

		\begin{tabular}{ | l | l | }
			\hline
			$x$ & $f(x)$ \\ \hline
			$0.1$ & $10$ \\
			$0.01$ & $100$ \\
			$0.001$ & $1000$ \\
			$0.0001$ & $10000$ \\
			\hline
		\end{tabular}

		$$\lim_{x \to 0} \frac{1}{x} = \infty$$

\end{itemize}


\section{Производная}
\label{section:derivative}
\subsection{Понятие}
Скорость роста функции

<график>

$y = k*x + b$

$k = \frac{f(x + \Delta x) - f(x)}{\Delta x}$ -- скорость роста

$$f^\prime(x) = \lim_{\Delta x \to 0} \frac{f(x + \Delta x) - f(x)}{\Delta x}$$ -- производная функции $f(x)$ в точке $x$





\section{Обозначения}
\begin{tabular}{ | l | l | }
	\hline
	Обозначение & Значение \\ \hline
	$\to$ & Стремится \\
	\hline
\end{tabular}



\bibliographystyle{plain}
\bibliography{references}
\end{document}
